%%%%%%%%%%%%%%%%%%%%%%%%%%%%%%%%%%%%%%%%%%%%%%%%%%
\section{Introduction}
%%%%%%%%%%%%%%%%%%%%%%%%%%%%%%%%%%%%%%%%%%%%%%%%%%
\begin{itemize}
\item Advantage of LArTPC: total-absorption, fine-grain, ...
\item We are interested in application of next generation neutrino and nucleon decay (100 kt LArTPC)
\item $p\to K^+\nu$: LArTPC is expected to have much better physics sensitivity than WC (~300 MeV/c Kaon ID can not seen in WC)
\item P32 and GLACIER program.
\item Two types of R\&D: (1) Make bigger detector,  (2) Understand property of LArTPC for more reliable physics sensitivity estimation
\item Study using not cosmic,  not neutrino beam, but charged particle (Kaon, pion, proton) from beam is important
\item We have constructed 250L LArTPC
\item Oct/2010 beam test at J-PARC K1.1Br beamline: Collected Kaon, pion, proton, and electron data
\item Establish reconstruction, for pion, proton, and Kaon
\item Develop realistic simulation
\item By Comparing data and MC in detail, we can understand basic property of the LArTPC detector
\item Section 2 describes experimental setup, Section 3 describes collected data sample, Section 4 Simulated event, Section 5 pion, Section 6 proton,   , Section 5 Kaon  
\end{itemize}
