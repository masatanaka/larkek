%%
%% Copyright 2007, 2008, 2009 Elsevier Ltd
%%
%% This file is part of the 'Elsarticle Bundle'.
%% ---------------------------------------------
%%
%% It may be distributed under the conditions of the LaTeX Project Public
%% License, either version 1.2 of this license or (at your option) any
%% later version.  The latest version of this license is in
%%    http://www.latex-project.org/lppl.txt
%% and version 1.2 or later is part of all distributions of LaTeX
%% version 1999/12/01 or later.
%%
%% The list of all files belonging to the 'Elsarticle Bundle' is
%% given in the file `manifest.txt'.
%%

%% Template article for Elsevier's document class `elsarticle'
%% with numbered style bibliographic references
%% SP 2008/03/01
%%
%%
%%
%% $Id: elsarticle-template-num.tex 4 2009-10-24 08:22:58Z rishi $
%%
%%
%%\documentclass[preprint,12pt]{elsarticle}
%%\documentclass[final,3p,times,twocolumn]{elsarticle}
\documentclass[final,3p,times]{elsarticle}

%% Use the option review to obtain double line spacing
%% \documentclass[preprint,review,12pt]{elsarticle}

%% Use the options 1p,twocolumn; 3p; 3p,twocolumn; 5p; or 5p,twocolumn
%% for a journal layout:
%% \documentclass[final,1p,times]{elsarticle}
%% \documentclass[final,1p,times,twocolumn]{elsarticle}
%% \documentclass[final,3p,times]{elsarticle}
%% \documentclass[final,3p,times,twocolumn]{elsarticle}
%% \documentclass[final,5p,times]{elsarticle}
%% \documentclass[final,5p,times,twocolumn]{elsarticle}

%% if you use PostScript figures in your article
%% use the graphics package for simple commands
%% \usepackage{graphics}
%% or use the graphicx package for more complicated commands
%% \usepackage{graphicx}
%% or use the epsfig package if you prefer to use the old commands
%% \usepackage{epsfig}

%% The amssymb package provides various useful mathematical symbols
\usepackage{amssymb}
%% The amsthm package provides extended theorem environments
%% \usepackage{amsthm}

%% The lineno packages adds line numbers. Start line numbering with
%% \begin{linenumbers}, end it with \end{linenumbers}. Or switch it on
%% for the whole article with \linenumbers after \end{frontmatter}.
%% \usepackage{lineno}

%% natbib.sty is loaded by default. However, natbib options can be
%% provided with \biboptions{...} command. Following options are
%% valid:

%%   round  -  round parentheses are used (default)
%%   square -  square brackets are used   [option]
%%   curly  -  curly braces are used      {option}
%%   angle  -  angle brackets are used    <option>
%%   semicolon  -  multiple citations separated by semi-colon
%%   colon  - same as semicolon, an earlier confusion
%%   comma  -  separated by comma
%%   numbers-  selects numerical citations
%%   super  -  numerical citations as superscripts
%%   sort   -  sorts multiple citations according to order in ref. list
%%   sort&compress   -  like sort, but also compresses numerical citations
%%   compress - compresses without sorting
%%
%% \biboptions{comma,round}

% \biboptions{}


\journal{NIMA}

\begin{document}

\begin{frontmatter}

%% Title, authors and addresses

%% use the tnoteref command within \title for footnotes;
%% use the tnotetext command for the associated footnote;
%% use the fnref command within \author or \address for footnotes;
%% use the fntext command for the associated footnote;
%% use the corref command within \author for corresponding author footnotes;
%% use the cortext command for the associated footnote;
%% use the ead command for the email address,
%% and the form \ead[url] for the home page:
%%
%% \title{Title\tnoteref{label1}}
%% \tnotetext[label1]{}
%% \author{Name\corref{cor1}\fnref{label2}}
%% \ead{email address}
%% \ead[url]{home page}
%% \fntext[label2]{}
%% \cortext[cor1]{}
%% \address{Address\fnref{label3}}
%% \fntext[label3]{}

\title{Performance of a 250L liquid Argon TPC
for sub-Gev charged particle identification}

%% use optional labels to link authors explicitly to addresses:
%% \author[label1,label2]{<author name>}
%% \address[label1]{<address>}
%% \address[label2]{<address>}

\author{J-PARC T32 collaboration}

\address{}

\begin{abstract}
%% Text of abstract

We have constructed a Liquid Argon time projection chamber (LArTPC) detector with 
fiducial mass of ~150 kg (250L Detector) as a part of the R\&D program 
of the next generation neutrino and nucleon decay detector.

This paper describes a study of particle identification performance of the 250L Detector
using well-defined charged particles (pions, kaons, and protons) with momentum of $\sim$800 MeV/$c$
obtained at J-PARC K1.1Br beamline.

\end{abstract}

\begin{keyword}
%% keywords here, in the form: keyword \sep keyword

%% MSC codes here, in the form: \MSC code \sep code
%% or \MSC[2008] code \sep code (2000 is the default)

\end{keyword}

\end{frontmatter}

%%
%% Start line numbering here if you want
%%
% \linenumbers


\section{Introduction}
\section{Experimental Apparatus}
\subsection{K1.1Br Beamline}
\subsection{250L Detector}
\subsection{Beamline Equipment}
\subsection{Oct/2010 Beam Test}
\section{Data Sample}
\subsection{Collected Data}
\subsection{Channel-by-Channel Calibration}
\subsection{Liquid Argon Purity}
\subsection{(``Cross Talk'')}
\section{Simulation Sample}
\section{Pion Result}
\section{Proton Result}
\section{Kaon Result}
\section{(Recombination)}
\section{Summary}

%% The Appendices part is started with the command \appendix;
%% appendix sections are then done as normal sections
%% \appendix

%% \section{}
%% \label{}

%% References
%%
%% Following citation commands can be used in the body text:
%% Usage of \cite is as follows:
%%   \cite{key}         ==>>  [#]
%%   \cite[chap. 2]{key} ==>> [#, chap. 2]
%%

%% References with bibTeX database:

\bibliographystyle{elsarticle-num}
\bibliography{<your-bib-database>}

%% Authors are advised to submit their bibtex database files. They are
%% requested to list a bibtex style file in the manuscript if they do
%% not want to use elsarticle-num.bst.

%% References without bibTeX database:

\begin{thebibliography}{00}

%% \bibitem must have the following form:
%%   \bibitem{key}...
%%

% \bibitem{}
%\cite{Araoka:2011pw}
\bibitem{Araoka:2011pw}
  O.~Araoka {\it et al.},
  %``A tagged low-momentum kaon test-beam exposure with a 250L LAr TPC (J-PARC
  %T32),''
  J.\ Phys.\ Conf.\ Ser.\  {\bf 308}, 012008 (2011)
  [arXiv:1105.5818 [physics.ins-det]].
  %%CITATION = 00462,308,012008;%%

\bibitem{Mihara:2004ft}
S.~Mihara [MEG Collaboration],
%``R&D work on a liquid-xenon photon detector for MEG experiment at PSI,''
Nucl.\ Instrum.\ Meth.\ A {\bf 518}, 45 (2004).
%%CITATION = NUIMA,A518,45;%%

%\cite{658352}
\bibitem{658352} 
  S.~Amoruso {\it et al.} [ICARUS Collaboration],
  %``Study of electron recombination in liquid argon with the ICARUS TPC,''
  Nucl.\ Instrum.\ Meth.\ A\ {\bf 523}, 275  (2004).
  %%CITATION = NUIMA,A523,275;%%

%\cite{649233}
\bibitem{649233} 
  S.~Amoruso, M.~Antonello, P.~Aprili, F.~Arneodo, A.~Badertscher, B.~Baibusinov, M.~Baldo-Ceolin and G.~Battistoni {\it et al.},
  %``Analysis of the liquid argon purity in the ICARUS T600 TPC,''
  Nucl.\ Instrum.\ Meth.\ A\ {\bf 516}, 68  (2004).
  %%CITATION = NUIMA,A516,68;%%

\bibitem{purity}
  A.~Bettini {\it et al.}, Nucl.\ Instrum.\ Meth.\ A\ {\bf 305}, 177 (1991).

\bibitem{3069654}
  P.V.C Hough 'Method and means for recognizing complex patterns',United States Patent Office 3069654(1962) 

\end{thebibliography}


\end{document}

%%
%% End of file `elsarticle-template-num.tex'.
